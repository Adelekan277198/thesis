\chapter{Methodology}
\section{Research design and approach}
The research design for this study is primarily exploratory and descriptive, aiming to systematically trace the historical evolution of place names within the geographical region corresponding to the transformation of Edo into Benin City. The study adopts a longitudinal perspective, examining place name changes across different historical periods, including pre-colonial, colonial, and post-colonial eras. The research design incorporates both qualitative and quantitative methods to analyze the linguistic, historical, and socio-cultural dimensions of place naming practices.
\subsection{Approach}
The research approach is interdisciplinary, drawing upon insights from linguistics, history, geography, and cultural studies to comprehensively examine the evolution of place names. The approach involves the following key components:

\begin{itemize}
    \item \textbf{Historical Research:} Archival research and historical analysis are conducted to reconstruct the historical context surrounding the transformation of Edo into Benin City. Primary sources such as colonial records, historical maps, and official documents are consulted to trace the origins and evolution of place names over time.

    \item \textbf{Linguistic Analysis:} Linguistic analysis is employed to investigate the phonological, morphological, and semantic features of place names. Etymological research and linguistic comparison are conducted to discern linguistic patterns, language borrowings, and cultural influences shaping place naming practices.

    \item \textbf{Geospatial Analysis:} Geographic Information System (GIS) technology is utilized to map and analyze the spatial distribution of place names within the study area. GIS enables spatial visualization, spatial analysis, and spatial interpolation of place names, allowing for the identification of spatial patterns and relationships.

\item \textbf{Natural Language Processing (NLP):} 	

    \item \textbf{Community Engagement:} Community engagement and stakeholder consultation are integral components of the research approach. Local communities, indigenous groups, and cultural experts are engaged in the research process to provide insights, perspectives, and indigenous knowledge related to place naming practices.

    \item \textbf{Ethical Considerations:} Ethical considerations, including cultural sensitivity and community consent, are carefully addressed throughout the research process. Respect for local knowledge systems, protection of cultural heritage, and informed consent protocols are prioritized in all research activities involving community engagement and data collection.
\end{itemize}

\subsection{Data collection methods (archival research, interviews, surveys)}
Data collection for studying the evolution of place names accompanying the transformation of Edo into Benin City involves a combination of archival research, interviews, and surveys. These methods are utilized to gather primary and secondary data sources that provide insights into the historical, linguistic, and socio-cultural dimensions of place naming practices.

Archival research is a primary method for accessing historical documents, maps, and official records relevant to the study area. Researchers consult archival repositories, such as national archives, libraries, and historical societies, to retrieve primary source materials dating back to the pre-colonial, colonial, and post-colonial periods. These archival documents include colonial gazettes, administrative reports, land surveys, census records, and historical maps that provide valuable information on place names, their origins, and changes over time.

Semi-structured interviews are conducted with key informants, local residents, community leaders, and cultural experts to gather qualitative data on place naming practices, indigenous knowledge, and cultural perspectives. Interview participants are selected based on their expertise in local history, language, and cultural heritage. Interview questions focus on topics such as the meanings and origins of place names, historical events influencing place naming, and community perceptions of cultural identity. Interviews provide rich, contextualized insights into the socio-cultural dynamics shaping place naming practices within the study area.

Surveys are administered to a representative sample of residents within the study area to collect quantitative data on place name usage, preferences, and perceptions. Surveys may be distributed in-person or electronically, using platforms such as online survey tools or mobile applications. Survey questions assess respondents' awareness of place names, preferences for traditional or colonial names, and perceptions of cultural identity associated with place names. Survey data provide quantitative insights into public attitudes and preferences regarding place naming practices, complementing qualitative findings from archival research and interviews.
\subsection{IT tools and techniques (GIS, NLP, digitization)}
Information Technology (IT) tools and techniques play a crucial role in studying the evolution of place names accompanying the transformation of Edo into Benin City. This section outlines the utilization of Geographic Information System (GIS), Natural Language Processing (NLP), and digitization methods in the research process.
\subsection*{Geographic Information System (GIS)}
GIS technology is employed to analyze and visualize spatial data related to place names within the study area. GIS software allows researchers to create digital maps, overlay historical maps with current geographical features, and conduct spatial analysis of place name distributions. Key functionalities of GIS in this study include:
Spatial Mapping: GIS enables researchers to map the spatial distribution of place names, identifying clusters, patterns, and spatial relationships between place names and geographical features.
Spatial Analysis: GIS facilitates spatial analysis techniques, such as proximity analysis, spatial interpolation, and hotspot analysis, to identify spatial trends and associations in place naming practices.
Spatial Visualization: GIS provides tools for visualizing place name data through interactive maps, charts, and graphs, enhancing the communication and interpretation of spatial patterns and trends.
\subsection*{Natural Language Processing (NLP)}
NLP techniques are utilized to analyze textual data sources, including historical documents, archival records, and linguistic corpora related to place names. NLP methods enable researchers to extract, analyze, and interpret textual information, revealing linguistic patterns, semantic meanings, and cultural contexts of place names. Key NLP techniques in this study include:
Text Mining: NLP algorithms are employed to extract relevant information from textual data sources, identifying place names, linguistic patterns, and contextual information embedded within historical documents.
Sentiment Analysis: NLP techniques analyze the sentiment and tone of textual data, identifying positive or negative associations with place names and cultural perceptions of identity.

\textbf{Topic Modeling:} NLP methods such as topic modeling are applied to identify themes, topics, and semantic clusters within textual data, uncovering underlying patterns and trends in place naming practices.

\textbf{Digitization}: methods are utilized to convert analog materials, such as historical maps, archival documents, and linguistic records, into digital formats for analysis and preservation. Digitization facilitates the accessibility, storage, and manipulation of historical and linguistic data, enabling researchers to conduct digital humanities research and interdisciplinary analysis. Key aspects of digitization in this study include:
Digitization of Historical Maps: Historical maps are digitized to create georeferenced digital maps that can be overlaid with current geographic data layers, facilitating spatial analysis and visualization of place names.
Digitization of Archival Documents: Archival documents, including colonial records, administrative reports, and linguistic texts, are digitized to create searchable digital archives, enabling researchers to extract textual data for NLP analysis and historical research.
Preservation and Access: Digitization ensures the long-term preservation and accessibility of historical and linguistic materials, safeguarding cultural heritage and enabling interdisciplinary research collaboration.
In summary, Geographic Information System (GIS), Natural Language Processing (NLP), and digitization methods are integral IT tools and techniques employed in the study of place name evolution in Edo and Benin City. These technologies enable researchers to analyze spatial, textual, and historical data sources, providing comprehensive insights into the socio-cultural dynamics of place naming practices within the study area.
\subsection{Ethical considerations and research limitations}
Ethical considerations and research limitations are important aspects to address in any scholarly study, including the investigation of the evolution of place names accompanying the transformation of Edo into Benin City. This section outlines the ethical principles guiding the research process and acknowledges the potential limitations inherent in the study.
\subsection{Ethical Considerations}
Respect for Cultural Sensitivity: I approached the study with sensitivity to the cultural heritage and identity of local communities. Respectful engagement with indigenous knowledge holders, cultural experts, and community stakeholders is essential to ensure that the research process respects and preserves cultural traditions.
Informed Consent: Informed consent is obtained from research participants, including interviewees, survey respondents, and contributors of archival materials. Participants are provided with clear information about the purpose of the study, the nature of their involvement, and their rights to confidentiality and privacy.
Protection of Cultural Heritage: I prioritize the protection and preservation of cultural heritage throughout the research process. Archival materials, historical documents, and indigenous knowledge are handled with care and respect, ensuring that cultural artifacts are not exploited or misrepresented.

\textbf{Community Engagement:} I engaged with local communities and stakeholders via Zoom Video in a collaborative and inclusive manner. Community input and feedback are solicited throughout the research process to ensure that the study reflects the perspectives and priorities of the communities being studied.
Data Confidentiality: Measures are implemented to safeguard the confidentiality and anonymity of research participants. Personal information collected during interviews, surveys, and archival research is kept confidential and used only for research purposes.
\subsection*{Research Limitations}
\textbf{Availability of Historical Records}: The availability and accessibility of historical records and archival materials posed limitations on the scope and depth of the research. Some historical documents may be incomplete, fragmented, or inaccessible, impacting the comprehensiveness of the study.


\textbf{Language Barriers:} Language barriers may present challenges in accessing and interpreting historical documents and linguistic sources. Translation services and linguistic expertise may be required to overcome language barriers and ensure accurate analysis of textual data.

\textbf{Bias and Interpretation:} Researchers must be mindful of potential biases in historical narratives and archival records. Historical accounts may reflect colonial perspectives or cultural biases, requiring critical interpretation and contextualization within broader socio-political contexts.

\textbf{Scope and Generalizability:} The study's findings may be limited in scope and may not be fully generalizable to other contexts beyond the specific geographical and historical context of Edo and Benin City. Researchers acknowledge the unique historical, cultural, and linguistic dynamics of the study area and interpret findings accordingly.

\textbf{Resource Constraints:} Resource constraints, including time, funding, and access to technology, may impact the research process and the ability to conduct comprehensive data collection and analysis. Researchers must navigate these constraints while striving to maintain rigor and integrity in the research design and execution.
