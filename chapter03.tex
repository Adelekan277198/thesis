\chapter{Methodology}

\section{Research Design and Approach}

The research design for this study is primarily exploratory and descriptive, aiming to systematically trace the historical evolution of place names within the geographical region corresponding to the transformation of Edo into Benin City. The study adopts a longitudinal perspective, examining place names across different historical periods, including pre-colonial, colonial, and post-colonial eras. The research design provides an analysis of the historical and socio-cultural dimensions of place naming practices.

\subsection{Approach}

The research approach is interdisciplinary, drawing upon insights from history, geography, and cultural studies to comprehensively examine the evolution of place names. This involves several key components. First, the historical documents and chronicles that are stored digitally are examined to perceive the historical background surrounding the transformation of Edo into Benin City. Primary sources such as colonial records, historical maps, and official documents are consulted to trace the origins and features of place names.

Second, this simple project was done with the aid of local historians and community leaders in the collection and validation of data. Their participation was crucial in ensuring the accuracy and cultural relevance of the information gathered. By engaging with these stakeholders, the research was able to incorporate a wide range of perspectives and validate the historical and cultural narratives embedded in the place names. This collaboration also ensured that the project adhered to ethical standards, respecting the cultural heritage and knowledge systems of the local communities.

\subsection{Data Acquisition Techniques}

The methodology of this study primarily involves the development of a relational data model and a web application to manage place names data. The data acquisition process includes integrating public APIs, such as OpenStreetMap API, and manual data entry based on gathered information from historical documents such as "The Benin Kingdom Document" and the ancient chronicles. 

The data model captures various attributes of place names, including administrative details, historical data, geographical features, and alternative names. The web application, developed using PHP, HTML, CSS, JavaScript, and AJAX, provides a user-friendly interface for data exploration and management. The data acquisition techniques employed include API integration, manual data entry, and data import. The OpenStreetMap API has been incorporated to retrieve location maps, and custom APIs have been created to enable bi-directional data flow, allowing users to view existing data and also add new entries. This ensures that data is stored in a~consistent and structured format. 


\subsection{Data Preprocessing Techniques}

The data preprocessing techniques employed include data cleaning, data transformation, data integration, data validation, and data enrichment.

Data cleaning involves duplicate removal, missing value handling, normalization, and error correction. Duplicate records are identified and merged using duplicate detection algorithms. Missing values are addressed by imputing with mean, median, or mode for numerical data and using placeholders or default values for categorical data. Data formats are standardized to ensure consistency across the database, and automated scripts and manual review processes are implemented to detect and correct common data entry errors.

Data transformation involves normalizing data to fit the relational database schema, ensuring each piece of data is stored in the appropriate table and field.  Data integration involves merging data from multiple sources, resolving conflicts and discrepancies through predefined rules and priority settings, and aligning data from different sources to the existing database schema, ensuring proper field mapping and data type conversion.

Data validation ensures that all data conforms to the database schema constraints and applies business rules to validate data, such as checking the range of valid postal codes. Referential integrity is ensured by validating that all foreign key references are consistent with the related tables. Data enrichment enhances the dataset by integrating additional data from external sources and adding metadata and annotations, such as source information, data reliability scores, and update timestamps.


\subsection{Ethical Considerations and Research Limitations}

Ethical considerations and research limitations are important aspects to address in any scholarly study, including the investigation of the evolution of place names accompanying the transformation of Edo into Benin City. This section outlines the ethical principles guiding the research process and acknowledges the potential limitations inherent in the study, specifically focusing on the use of publicly available data and direct community engagement.

\subsubsection{Ethical Considerations}

Respect for cultural sensitivity is paramount. The study is approached with sensitivity to the cultural heritage and identity of local communities. Respectful engagement with indigenous knowledge holders, and cultural experts is essential to ensure that the research process respects and preserves cultural traditions. 

The protection of cultural heritage is prioritized throughout the research process. Archival materials, historical documents, and indigenous knowledge are handled with care and respect, ensuring that cultural artifacts are not exploited or misrepresented. Community input and feedback are solicited throughout the research process to ensure that the study reflects the perspectives and priorities of the communities being studied. 

\subsubsection{Research Limitations}

The availability and accessibility of historical records and materials posed limitations on the scope and depth of the research. Some historical documents may be incomplete, fragmented, or inaccessible, impacting the comprehensiveness of the study. Language barriers may present challenges in accessing and interpreting historical documents. Translation services and linguistic expertise may be required to overcome language barriers and ensure accurate analysis of textual data.

Researchers must be mindful of potential biases in historical narratives and archival records. Historical accounts may reflect colonial perspectives or cultural biases, requiring critical interpretation and contextualization within broader socio-political contexts. The study's findings may be limited in scope and may not be fully generalizable to other contexts beyond the specific geographical and historical context of Edo and Benin City. Researchers acknowledge the unique historical, cultural, and socio-political dynamics of the study area and interpret findings accordingly.

Resource constraints, including time, funding, and access to technology, may impact the research process and the ability to conduct comprehensive data collection and analysis. Researchers must navigate these constraints while striving to maintain rigor and integrity in the research design and execution.
