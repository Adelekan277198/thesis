%\pdfbookmark[0]{Abstract}{abstract.1}
%\phantomsection
%\addcontentsline{toc}{chapter}{Abstract}
%%% The following was not used (i.e. the creation of an unnumbered chapter for the abstract was abandoned) 
%%%\begingroup
%%%\setlength\beforechapskip{48pt} % for some reason there was a slight difference in the position of the numbered and unnumbered chapter headers 
%%%\chapter*{\centering Abstrakt}
%%%\endgroup
%%%\label{sec:abstrakt}
%%%Lorem ipsum dolor sit amet eleifend et, congue arcu. Morbi tellus sit amet, massa. Vivamus est id risus. Sed sit amet, libero. Aenean ac ipsum. Mauris vel lectus. 
%%%
%%%Nam id nulla a adipiscing tortor, dictum ut, lobortis urna. Donec non dui. Cras tempus orci ipsum, molestie quis, lacinia varius nunc, rhoncus purus, consectetuer congue risus. 
%\mbox{}\vspace{2cm} % can be shifted depending on the length of the abstract 
\begin{abstract}
This thesis examines the historical development of place names in the area where Edo became Benin City using information technology (IT) tools. An extensive digital database is created by digitizing old maps and documents. The work reveals trends among historical records, colonial influences, and socio-political dynamics driving place name changes by integrating GIS spatial analysis with NLP textual analysis. Through IT-enabled analysis, this interdisciplinary technique clarifies the renaming of geographical features, adding to historical and urban studies.
\end{abstract}
\mykeywords
% It would be good to copy the keywords to the metadata of the pdf document (in the file Thesis.tex)
% Unfortunately, the implemented macro does not do it automatically, so the manual copy remains. 

{
\selectlanguage{polish}
\begin{abstract}
Praca bada historyczną ewolucję nazw miejscowości od Edo do Benin City przy użyciu zaawansowanej technologii. Analizuje dokumenty historyczne, mapy i zapisy, aby stworzyć wizualne przedstawienie zmian nazw, biorąc pod uwagę czynniki kulturowe, historyczne i społeczno-ekonomiczne. Celem projektu jest integracja dokumentów historycznych z nowoczesną technologią w celu całościowego zrozumienia przemian nazw miejscowości w Beninie.
\end{abstract}
\mykeywords 
}
