%\pdfbookmark[0]{Abstract}{abstract.1}
%\phantomsection
%\addcontentsline{toc}{chapter}{Abstract}
%%% The following was not used (i.e. the creation of an unnumbered chapter for the abstract was abandoned) 
%%%\begingroup
%%%\setlength\beforechapskip{48pt} % for some reason there was a slight difference in the position of the numbered and unnumbered chapter headers 
%%%\chapter*{\centering Abstrakt}
%%%\endgroup
%%%\label{sec:abstrakt}
%%%Lorem ipsum dolor sit amet eleifend et, congue arcu. Morbi tellus sit amet, massa. Vivamus est id risus. Sed sit amet, libero. Aenean ac ipsum. Mauris vel lectus. 
%%%
%%%Nam id nulla a adipiscing tortor, dictum ut, lobortis urna. Donec non dui. Cras tempus orci ipsum, molestie quis, lacinia varius nunc, rhoncus purus, consectetuer congue risus. 
%\mbox{}\vspace{2cm} % can be shifted depending on the length of the abstract 
\begin{abstract}
This thesis explores the historical development of place names in the region transitioning from Edo to Benin City, focusing on the digitization and analysis of historical documents and maps based on the available data set. The study creates a digital database by leveraging data acquisition techniques. It provides an overview of colonial influences and socio-political dynamics impacting place name changes. The research involves the development of a data model and an information system to manage place name data, including the preservation of pronunciations through recorded voices, enhancing historical and urban studies.
\end{abstract}
\mykeywords
% It would be good to copy the keywords to the metadata of the pdf document (in the file Thesis.tex)
% Unfortunately, the implemented macro does not do it automatically, so the manual copy remains. 

{
\selectlanguage{polish}
\begin{abstract}
Praca bada historyczn¹ ewolucjê nazw miejscowoœci od Edo do Benin City przy u¿yciu zaawansowanej technologii. Analizuje dokumenty historyczne, mapy i zapisy, aby stworzyæ wizualne przedstawienie zmian nazw, bior¹c pod uwagê czynniki kulturowe, historyczne i spo³eczno-ekonomiczne. Celem projektu jest integracja dokumentów historycznych z nowoczesn¹ technologi¹ w celu ca³oœciowego zrozumienia przemian nazw miejscowoœci w Beninie.
\end{abstract}
\mykeywords 
}
