%\pdfbookmark[0]{Abstract}{abstract.1}
%\phantomsection
%\addcontentsline{toc}{chapter}{Abstract}
%%% The following was not used (i.e. the creation of an unnumbered chapter for the abstract was abandoned) 
%%%\begingroup
%%%\setlength\beforechapskip{48pt} % for some reason there was a slight difference in the position of the numbered and unnumbered chapter headers 
%%%\chapter*{\centering Abstrakt}
%%%\endgroup
%%%\label{sec:abstrakt}
%%%Lorem ipsum dolor sit amet eleifend et, congue arcu. Morbi tellus sit amet, massa. Vivamus est id risus. Sed sit amet, libero. Aenean ac ipsum. Mauris vel lectus. 
%%%
%%%Nam id nulla a adipiscing tortor, dictum ut, lobortis urna. Donec non dui. Cras tempus orci ipsum, molestie quis, lacinia varius nunc, rhoncus purus, consectetuer congue risus. 
%\mbox{}\vspace{2cm} % can be shifted depending on the length of the abstract 
\begin{abstract}
This thesis investigates the historical development of place names during the transformation of Edo into Benin City. The research emphasizes the digitization and analysis of historical documents and maps, all of which are publicly available and sourced from various sources, interviews with local historians and community leaders, and surveys. The study creates a comprehensive digital database by leveraging these publicly accessible data acquisition techniques. It provides an overview of colonial influences and socio-political dynamics impacting place name changes. The research involves the development of a data model and an information system to manage place name data, including the preservation of pronunciations through recorded voices, enhancing historical and urban studies.
\end{abstract}
\mykeywords
% It would be good to copy the keywords to the metadata of the pdf document (in the file Thesis.tex)
% Unfortunately, the implemented macro does not do it automatically, so the manual copy remains. 

{
\selectlanguage{polish}
\begin{abstract}
   Niniejsza rozprawa bada historyczny rozwój nazw miejsc podczas transformacji Edo w Benin City. Badania kładą nacisk na digitalizację i analizę historycznych dokumentów i map, które są publicznie dostępne i pochodzą z Internetu, wywiady z lokalnymi historykami i liderami społeczności oraz ankiety. Badanie tworzy kompleksową cyfrową bazę danych, wykorzystując te publicznie dostępne techniki pozyskiwania danych. Zapewnia przegląd wpływów kolonialnych i dynamiki społeczno-politycznej wpływającej na zmiany nazw miejsc. Badania obejmują opracowanie modelu danych i systemu informacyjnego do zarządzania danymi o nazwach miejsc, w tym zachowanie wymowy za pomocą nagranych głosów, co wzmacnia badania historyczne i miejskie.
\end{abstract}
\mykeywords 
}
